\documentclass{jarticle}

\usepackage{setspace} % setspaceパッケージのインクルード
\usepackage[top=50truemm,bottom=50truemm,left=30truemm,right=30truemm]{geometry}
\usepackage{url}

%\setstretch{1.75} % ページ全体の行間を設定

\begin{document}
\begin{titlepage}
\title{ソフトウェア演習II 問題8}
\date{2016/11/16}
\author{14S1020U 小林佐保}
\maketitle
\thispagestyle{empty}
\end{titlepage}
\section{解いた問題}

\begin{description}
 \item[1.] 12章の練習問題を、1~17から2つ以上(似てないものを)選んで解け。
 \begin{description}
  \item[1.] 基本手続き*の追加
  \item[2.] 基本手続き/の追加
  \item[3.] 基本手続き-の改良
  \item[4.] 基本手続き-の改良
  \item[8.] 基本手続きlist?の追加
 \end{description}
 \item[2.18.] 12章の練習問題を、18~26、31、32から1つ以上選んで解け。(文字列型の追加)
 \item[3.27.] 12章の練習問題の27を解け。(defineの改良。)
 \item[5.29.] 12章の練習問題の29を解け。(letの実現。)
\end{description}

\section{プログラム解説書}
\subsection{1.1. 基本手続き*の追加}
\subsubsection{ts:intern-primitive-proceduresに*を追加}
これによって、(ts)実行時に'*と関数tsp:*、引数の数2の情報が基本手続き*として登録される。\par
\subsubsection{tsp:*の定義}
TSのintegerとして与えられた引数からschemeの値を取り出し、それらの引数をschemeの*に与えて評価した値をTSのintegerに直す。これをこの関数を評価した値とする。\par

\subsection{1.2. 基本手続き/の追加}
\subsubsection{ts:intern-primitive-proceduresにを追加}
これによって、(ts)実行時に'/と関数tsp:/、引数の数2の情報が基本手続き/として登録される。\par
\subsubsection{tsp:*の定義}
TSのintegerとして与えられた引数からschemeの値を取り出し、それらの引数をschemeのquotientに与えて評価した値をTSのintegerに直す。これをこの関数を評価した値とする。schemeの関数quotientを評価した値は整数割り算の商であり、これは実数の範囲で割り算をし、小数点を切り捨てた値と同じである。\par

\subsection{1.3. 基本手続き-の改良}
\subsubsection{ts:intern-primitive-proceduresの-を変更}
これによって、(ts)実行時に登録される基本手続き-の引数の数を2から'anyに変更できる。\par
\subsubsection{tsp:-の定義}
再帰呼び出しによって和を求める関数loop-sumを用意し、それを利用して、第一引数から第二引数以降をloop-sumに渡して得た和を引いた値をtsp:-を評価した値とする。\par

\subsection{1.4. 基本手続き-の改良}
\subsubsection{tsp:-の定義}
引数がひとつかどうかを判定し、二つ以上であれば1.3.のプログラムを実行し、ひとつであれば、その引数から得たschemeの値をschemeの-関数に渡した値をTSのintegerに直し、この関数を評価した値とする。

\subsection{1.8. 基本手続きlist?の追加}
\subsubsection{ts:intern-primitive-proceduresにlist?を追加}
これによって、(ts)実行時に'LIST?と関数tsp:list?、引数の数1の情報が基本手続きlist?として登録される。\par
\subsubsection{tsp:list?の定義}
再帰呼び出しによってcdrを辿り、空リスト'()に行き着けばlistである、そうでなければlistでないとする。\par

\subsection{2.18. 文字列型の追加}
\subsubsection{ts:make-string関数}
schemeの値とSTRINGタグとのペアを作る。\par
\subsubsection{ts:string?関数}
STRINGタグがついていればstringであると判定する。\par
\subsubsection{ts:scheme-obj->ts-objにstringを追加}
読み込み等するときにschemeのstring型であればTSのstring型に変換するように追加。\par
\subsubsection{ts:print-valueにstringを追加}
TSのstring型は値をそのまま表示するようにifの条件に追加。\par
\subsubsection{ts:evalにstringを追加}
string型を評価した場合それ自身を返すように追加。\par
\subsubsection{ts:intern-primitive-proceduresに代表的な文字列を扱う基本手続きを追加}
今回はstring-length、string=?、string-appendを追加。
\subsubsection{tsp:string-lengthの定義}
schemeのstring-lengthを利用して実装。\par
\subsubsection{tsp:string=?の定義}
schemeのstring=?を利用して実装。\par
\subsubsection{tsp:string-appendの定義}
schemeのstring-appendを利用して実装。\par

\subsection{3.27. defineの改良}
\subsubsection{ts:do-special-form関数}
(DEFINE)のcaseを編集。ts:do-special-formに渡された引数のリストの2つめの要素(以下defineの第一引数と呼ぶ)がペア(要素1以上のリストもしくはペア)であれば、簡略記法であると判断する。その場合は、適切に引数の順序を入れ替え、等価な通常の記法の形に直してdefine-varに渡す。TSのlambdaはそれ自身独立した関数として定義されていなかったので、ts:make-compound-procedureを利用した。間略記法でない場合は以前の通りに評価する。\par

\subsection{5.29. letの追加}
\subsubsection{ts:map関数}
letを実装するために、関数とリストを引数にとり、リストの各項目に関数を適応して得られた値をリストにして返すmap関数を実装。\par
\subsubsection{ts:special-form?関数に追加}
letを特殊形式として登録。\par
\subsubsection{ts:do-special-form関数}
letに与えられた引数をmap関数を利用して適切に並べ替え、3.27.のdefineの改良同様、lambdaの代わりのts:make-compound-procedureに渡した。\par


\end{document}
